\chapter{Preface}
\label{sec:intro}

%\section{A brief overview of libRadtran}

{\sl libRadtran} is a library of radiative transfer routines and
programs.  The central program of the {\sl libRadtran} package is the
radiative transfer tool {\sl uvspec}.  {\sl uvspec} was originally
designed to calculate spectral irradiance and actinic flux in the
ultraviolet and visible parts of the spectrum \citep{kylling92a} where
the name stems from. Over the years, {\sl uvspec} has undergone
numerous extensions and improvements. {\sl uvspec} now includes the
full solar and thermal spectrum, currently from 120~nm to
100~$\mu$m. It has been designed as a user-friendly and versatile tool
which provides a variety of options to setup and modify an atmosphere
with molecules, aerosol particles, water and ice clouds, and a surface
as lower boundary. One of the unique features of {\sl uvspec} is that
it includes not only one but a selection of about ten different
radiative transfer equation solvers, fully transparent to the user,
including the widely-used DISORT code by \citet{Stamnes1988c} and its
C-code version \citep{buras2011b}, a fast two-stream code
\citep{Kylling1995}, a polarization-dependent code polRadtran
\citep{Evans1991}, and the fully three-dimensional Monte
Carlo code for the physically correct tracing of photons in cloudy
atmospheres, MYSTIC \citep{mayer2009, emde2007, emde2010, buras2011a,
emde2011}. MYSTIC optionally allows to consider polarization and fully
spherical geometry. Please note that the public release includes only a
1D version of MYSTIC.

{\sl libRadtran} also provides related utilities, like e.g. a Mie program 
({\sl mie}), some utilities for the calculation of the position of the 
sun ({\sl zenith}, {\sl noon}, {\sl sza2time}), a few tools for 
interpolation, convolution, and integration ({\sl spline}, {\sl conv}, 
{\sl integrate}), and several other small tools for setting up
{\sl uvspec} input and postprocessing {\sl uvspec} output. 

Further general information about {\sl libRadtran} including examples
of use may be found in the
reference publication \citep{mayer2005}. 

It is expected that the reader is familiar with radiative transfer
terminology. In addition, a variety of techniques and
parameterizations from various sources are used. For more information
about the usefulness and applicability of these methods in a specific
context, the user is referred to the referenced literature.

{\sl Please note that this document is by no means complete. It is
under rapid development and major changes will take place.} 


\section*{Acknowledgements}

Many people have already contributed to {\sl libRadtran}'s
development. In addition to Bernhard Mayer (\email{bernhard.mayer (at)
  dlr.de}), Arve Kylling (\email{arve.kylling (at) gmail.com}), Claudia Emde
(\email{claudia.emde (at) lmu.de}), Robert Buras
(\email{robert.buras (at) lmu.de}), Josef Gasteiger
(\email{josef.gasteiger (at) lmu.de}), Bettina Richter
(\email{bettina.richter (at) lmu.de}) and Ulrich 
Hamann (\email{hamann (at) knmi.nl}) the following people have
contributed to {\sl libRadtran} or helped out in various other ways
(the list is almost certainly incomplete -- please let us know if we
forgot somebody):

\begin{itemize}
\item The \code{disort} solver was developed by Knut Stamnes, Warren Wiscombe, 
  S.C.~Tsay, and K.~Jayaweera
\item The translation from the FORTRAN version of the DISORT solver to C-code was performed by Timothy E.~Dowling
\item Warren Wiscombe provided the Mie code \code{MIEV0}, and the routines to calculate
  the refractive indices of water and ice, \code{REFWAT} and \code{ICEWAT}.
\item Seiji Kato (\email{kato (at) aerosol.larc.nasa.gov}) provided the 
  correlated-k tables described in Kato et al. (1999).
\item Tom Charlock (\email{t.p.charlock (at) larc.nasa.gov}), Quiang Fu 
  (\email{qfu (at) atm.dal.ca}), 
  and Fred Rose (\email{f.g.rose (at) larc.nasa.gov}) provided the most recent version 
  of the Fu and Liou code.
\item David Kratz (\email{kratz (at) aquila.larc.nasa.gov}) provided the routines 
  for the simulation of the AVHRR channels described in 
  Kratz (1995).
\item Frank Evans (\email{evans (at) nit.colorado.edu}) provided the
  \code{polradtran} solver.
\item Ola Engelsen provided data and support for different ozone 
absorption cross sections.
\item Albano Gonzales (\email{aglezf (at) ull.es}) included the Yang et
  al. (2000), Key et al. (2002) ice crystal parameterization. 
\item Tables for the radiative properties of ice clouds for different
  particle ``habits'' were obtained from Jeff Key and Ping Yang,
  Yang et al. (2000), Key et al. (2002). In addition, Ping Yang and 
  Heli Wei kindly provided a comprehensive database of particle 
  single scattering properties which we used to derive a consistent 
  set of ice cloud optical properties for the spectral range 0.2 - 100 micron 
  following the detailed description in Key et al. (2002).
  A comprehensive dataset including the full phase matrices has been 
  generated and provided by Hong Gang. 
\item Paul Ricchiazzi (\email{paul (at) icess.ucsb.edu}) and colleagues
  allowed us to include the complete gas absorption parameterization 
  of their model SBDART into {\sl uvspec}.
\item Luca Bugliaro (\email{luca.bugliaro (at) dlr.de}) wrote the analytical 
  TZS solver (thermal, zero scattering).
\item Sina Lohmann (\email{sina.lohmann (at) dlr.de}) reduced the ``overhead time''
  for reading the Kato et al. tables dramatically which resulted in 
  a speedup of a factor of 2 in a twostr solar irradiance calculation.
\item Detailed ice cloud properties were provided by Bryan Baum
  (\email{bryan.baum (at) ssec.wisc.edu}). 
\item Yongxiang Hu (\email{yongxiang.hu-1 (at) nasa.gov}) provided the delta-fit 
  program used to calculate the Legendre coefficients for \code{ic\_properties baum\_hufit}.
\item UCAR/Unidata for providing the \emph{netCDF} library.
\item Many unnamed users helped to improve the code by identifying 
  or fixing bugs in the code. 
\end{itemize}



